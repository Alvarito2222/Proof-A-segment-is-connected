\documentclass{article}
\usepackage[utf8]{inputenc}
\usepackage{amssymb} 
\usepackage{parskip} 

\title{Proof : The intersection of two open sets is open}
\author{aleonp000 }
\date{January 2023}

\begin{document}


\noindent
\textbf{Theorem}:  A segment is Connected.

\textit{Proof}: 
Suppose $(a,b)$ is not connected. Then $(a,b)$ = $H\cup K$ where $H$ and $K$ are non-empty disjoint mutually separated sets .
 

Consider $(a,z)$ , where $z$ is a variable. Consider $U =$ \textbraceleft$ z$ in $ (a,b) : (a,z) $ subset $H$\textbraceright . Let $W$ = l.u.b of $U$. We know that $W \leq b$ . 


$\hookrightarrow$ Case 1: $W = b$. 
If this happens $K$ will be empty. This contradicts that $K$ is non-empty.

$\hookrightarrow$ Case 2: $W < b$.

\textcircled{1} Claim:  $W$ is a limit point of $H$. Assume it's not. Then $\exists$ $\epsilon > 0$ s.t. $(W-\epsilon , W+ \epsilon )\cap H = \emptyset$ . Then $\exists$ smaller numbers than $W$ that are also upper bounds of $U$.

\textcircled{2} Claim: $W$ is a limit point of $K$. Assume it's not. Then $\exists$ $\epsilon > 0$ s.t. $(W-\epsilon , W+ \epsilon )\cap K = \emptyset$ . Since everything between $(W , W+\epsilon)$ is in $H($Part of $U)$ , $W$ itself would not be a l.u.b. or even upper bound. This is a contradiction.

WLOG , assume $W$ is in $H$,

Since $K$ is mutually separated from $H$ and $W$ is a limit point of $K$ and $H$ , this contradicts our first assumption that $(a,b)$ = $H\cup K$.





 \noindent 
 $\therefore$ A segment is Connected      \spaceskip1cm   \hfill                     $\square$
 

\end{document}
